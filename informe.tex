\documentclass[a4paper,twocolumn]{article}


\usepackage[sc]{mathpazo} % Use the Palatino font
\usepackage[T1]{fontenc} % Use 8-bit encoding that has 256 glyphs
\usepackage[utf8]{inputenc} % Use utf-8 as encoding
\linespread{1.05} % Line spacing - Palatino needs more space between lines
\usepackage{microtype} % Slightly tweak font spacing for aesthetics
\usepackage{graphicx}

\usepackage[spanish]{babel} % Language hyphenation and typographical rules
%\usepackage[galician]{babel} % Change to this if using galician

\usepackage[hmarginratio=1:1,top=32mm,columnsep=20pt]{geometry} % Document margins
\usepackage[hang, small,labelfont=bf,up,textfont=it,up]{caption} % Custom captions under/above floats in tables or figures
\usepackage{booktabs} % Horizontal rules in tables

\usepackage{enumitem} % Customized lists
\setlist[itemize]{noitemsep} % Make itemize lists more compact

\usepackage{abstract} % Allows abstract customization
\renewcommand{\abstractnamefont}{\normalfont\bfseries} % Set the "Abstract" text to bold
\renewcommand{\abstracttextfont}{\normalfont\small\itshape} % Set the abstract itself to small italic text

\usepackage{titlesec} % Allows customization of titles
\renewcommand\thesection{\Roman{section}} % Roman numerals for the sections
\renewcommand\thesubsection{\Alph{subsection}} % roman numerals for subsections
\titleformat{\section}[block]{\large\scshape\centering}{\thesection.}{1em}{} % Change the look of the section titles
\titleformat{\subsection}[block]{\large}{\thesubsection.}{1em}{} % Change the look of the section titles

\usepackage{fancyhdr} % Headers and footers
\pagestyle{fancy} % All pages have headers and footers
\fancyhead{} % Blank out the default header
\fancyfoot{} % Blank out the default footer
%\fancyhead[C]{Running title $\bullet$ May 2016 $\bullet$ Vol. XXI, No. 1} % Custom header text
\fancyfoot[C]{\thepage} % Custom footer text

\usepackage{titling} % Customizing the title section

\usepackage{hyperref} % For hyperlinks in the PDF

%----------------------------------------------------------------------------------------
%	TITLE SECTION
%----------------------------------------------------------------------------------------

\setlength{\droptitle}{-4\baselineskip} % Move the title up

\pretitle{\begin{center}\huge\bfseries} % Article title formatting
	\posttitle{\end{center}} % Article title closing formatting

\title{Sincronización de procesos con mútexes} % Article title
\author{%
	\textsc{Pablo Seijo García, Javier Pereira Romero} \\[1ex] % Your name
	\normalsize Sistemas Operativos II\\
	\normalsize Grupo 04 \\ % Your institution
	\normalsize \{pablo.garcia.seijo, javier.pereira.romero\}@rai.usc.es % Your email address
	%\and % Uncomment if 2 authors are required, duplicate these 4 lines if more
	%\textsc{Jane Smith}\thanks{Corresponding author} \\[1ex] % Second author's name
	%\normalsize University of Utah \\ % Second author's institution
	%\normalsize \href{mailto:jane@smith.com}{jane@smith.com} % Second author's email address
}
\date{\today} % Leave empty to omit a date
\renewcommand{\maketitlehookd}{%
	\begin{abstract}
		\noindent En este informe, abordaremos la resolución del problema del productor-consumidor mediante el uso de mútexes y variables de condición.  \\\mbox{}\\
		 \textbf{\textit{Palabras clave}: Hilos, mútexes, productor-consumidor, condición de carrera, variable de condición\ldots}
	\end{abstract}
}

%----------------------------------------------------------------------------------------

\begin{document}
	
	% Print the title
	\maketitle
	
	%----------------------------------------------------------------------------------------
	%	ARTICLE CONTENTS
	%----------------------------------------------------------------------------------------
	
	\section{Introducción}

        En el presente informe se describe una serie de experimentos diseñados para estudiar las condiciones de carrera existentes en el problema denominado problema del productor-consumidor.
        
        Para el estudio, se realizarán varios programas breves escritos en el lenguaje de bajo nivel C, ejecutados en un sistema operativo Linux. Para la resolución del problema del productor-consumidor se implementarán los mútexes y las variables de condición como mecanismos de control.

 
%	Introducción al problema tratado, incluyendo las referencias  necesarias. Por ejemplo: ``Este trabajo se basa en los estudios teóricos realizados en~\cite{Intel:2005} y \cite{spec}". En este apartado se plantean el problema a resolver, objetivos a alcanzar y metodología seguida para alcanzarlos.
	
%	Es una introducción al problema, no se desarrollarán los contenidos aquí. 
	
%	La introducción termina indicando en pocas palabras de qué secciones consta el resto del documento y de qué trata cada una. Se mencionan todas las secciones salvo la de referencias (bibliografía). 

	%------------------------------------------------
	
%-------------------------------------------

\section{Descripción de la Implementación y Experimentos}

En este apartado se describe la implementación de la solución al problema del productor-consumidor modificada para manejar múltiples productores y consumidores, utilizando un buffer LIFO y variables de condición para sincronizar el acceso.

\subsection{Estructura del Programa}
El programa se basa en la definición de un número de productores (\textit{P}) y consumidores (\textit{C}), que interactúan mediante un buffer de tamaño (\textit{N = 12}). Cada productor produce un total de 18 ítems, asegurándose que cada ítem producido sea consumido. El acceso al buffer se controla mediante mutexes para evitar condiciones de carrera, mientras que las variables de condición ayudan a sincronizar la producción y el consumo según la capacidad del buffer.

\subsection{Funcionamiento del Código}
Cada productor, tras producir un ítem, intenta colocarlo en el buffer. Si el buffer está lleno, el productor se bloquea hasta que un consumidor extraiga un ítem. Similarmente, cada consumidor espera a que haya ítems disponibles para consumir. Adicionalmente, tanto productores como consumidores participan en la suma de elementos pares o impares de un arreglo global \textit{T}, respectivamente.

\subsection{Parámetros de Ejecución}
El programa se ejecuta con diferentes tiempos de espera (sleeps) introducidos para simular variadas condiciones de carga y sincronización, especificados como argumentos al iniciar el programa. Esto incluye tiempos de espera para la producción, inserción en el buffer, y sumatoria de los elementos del arreglo.

\section{Resultados de los Experimentos}

Los experimentos se realizaron bajo dos configuraciones principales: una con 1 productor y 2 consumidores, y otra con 5 productores y 2 consumidores. Los resultados muestran cómo se comportan los productores y consumidores bajo diferentes cargas.

\subsection{Experimento con 3 Productores y 3 Consumidores}

Este experimento, con igual número de productores y consumidores, muestra un balance más estable entre producción y consumo. Los resultados indican menos variabilidad en los tiempos de espera para ambos productores y consumidores. Esto sugiere que un número equilibrado de productores y consumidores puede mejorar la eficiencia general del sistema, manteniendo el buffer en un estado de uso óptimo.

\subsection{Experimento con 5 Productores y 2 Consumidores}

En este experimento con más productores que consumidores, se observa una mejora notable en la disponibilidad de ítems en el buffer, lo que reduce el tiempo de espera para los consumidores. Sin embargo, la alta competencia entre los productores por acceder y modificar el arreglo \textit{T} puede introducir complejidades en la gestión del buffer. A pesar de esto, los consumidores manejan eficientemente el proceso de consumo y sumatoria de elementos impares, resaltando la efectividad de los semáforos y las variables de condición en la coordinación entre múltiples hilos.

\subsection{Experimento con 2 Productores y 5 Consumidores}

Este experimento demuestra cómo el sistema maneja una demanda de consumo más alta que la de producción. Los resultados muestran una rápida extracción de ítems del buffer por parte de los consumidores, lo que puede llevar a situaciones donde los productores tienen que esperar más frecuentemente a que haya espacio disponible para producir nuevos ítems. Aquí, los consumidores a menudo tienen que esperar a que los productores llenen el buffer, demostrando la eficacia de los semáforos en la regulación del acceso al buffer bajo condiciones de alta demanda de consumo. 

Un factor a comentar sería que en este caso nuestro programa al finalizar de realizar las operaciones no sale del programa, y podemos sacar conclusiones de que se producen interbloqueos en el programa.

%------------------------------------------------
	
\section{Conclusiones}

Este informe ha demostrado la efectividad de usar mutexes y variables de condición para manejar la sincronización entre múltiples productores y consumidores en el contexto de un buffer LIFO. Los experimentos sugieren que la configuración del número de productores y consumidores, junto con los tiempos de operación adecuadamente ajustados, es crucial para el equilibrio entre la producción y el consumo y la eficiencia de la suma de elementos del arreglo.

%----------------------------------------------------------------------------------------
%	Referencias
%----------------------------------------------------------------------------------------
	
\begin{thebibliography}{99} % Bibliografía - alternativamente, se recomienda el uso de bibtex o biblatex
		
	\bibitem[1]{Sistemas Operativos Modernos}
    Andrew S. Tanenbaum. 
	\newblock {\em Sistemas Operativos Modernos, 3ª edición.}
	\newblock Editorial Prentice-Hall, 2009, secciones 2.3.4 - 2.3.5.
%	\newblock \href{https://apps4two.com/curso_dba/bibliografia/2-Sistemas%20operativos%20moderno%203ed%20Tanenbaum.pdf}{https://apps4two.com/curso_dba/bibliografia/2Sistemasoperativosmoderno203ed20Tanenbaum.pdf}, 	\newblock [online]
		
\end{thebibliography}
	
	%----------------------------------------------------------------------------------------
	
\end{document}